% My colors
\definecolor{primary}{HTML}{4361ee}
\definecolor{primarylight}{HTML}{4895ef}
\definecolor{primarydark}{HTML}{3f37c9}
\definecolor{secondary}{HTML}{ff8500}
\definecolor{secondarylight}{HTML}{ff9e00}
\definecolor{secondarydark}{HTML}{ff6d00}
\definecolor{tertiary}{HTML}{5a189a}
\definecolor{tertiarylight}{HTML}{9d4edd}
\definecolor{tertiarydark}{HTML}{240046}

% Custom Theorem Styles
\mdfsetup{skipabove=10pt,skipbelow=0pt}

\declaretheoremstyle[
    headfont=\bfseries\color{primary},
    bodyfont=\normalfont,
    mdframed={
        linewidth=2pt,
        leftline=true, rightline=false, topline=false, bottomline=false,
        linecolor=primary, backgroundcolor=primary!5,
    }
]{primary}

\declaretheoremstyle[
    headfont=\bfseries\color{primarylight},
    bodyfont=\normalfont,
    mdframed={
        linewidth=2pt,
        leftline=true, rightline=false, topline=false, bottomline=false,
        linecolor=primarylight, backgroundcolor=primarylight!5,
    }
]{primarylight}

\declaretheoremstyle[
    headfont=\bfseries\color{primarydark},
    bodyfont=\normalfont,
    mdframed={
        linewidth=2pt,
        leftline=true, rightline=false, topline=false, bottomline=false,
        linecolor=primarydark, backgroundcolor=primarydark!5,
    }
]{primarydark}

\declaretheoremstyle[
    headfont=\bfseries\color{secondary},
    bodyfont=\normalfont,
    mdframed={
        linewidth=2pt,
        leftline=true, rightline=false, topline=false, bottomline=false,
        linecolor=secondary, backgroundcolor=secondary!5,
    }
]{secondary}

\declaretheoremstyle[
    headfont=\bfseries\color{secondarylight},
    bodyfont=\normalfont,
    mdframed={
        linewidth=2pt,
        leftline=true, rightline=false, topline=false, bottomline=false,
        linecolor=secondarylight, backgroundcolor=secondarylight!5,
    }
]{secondarylight}

\declaretheoremstyle[
    headfont=\bfseries\color{secondarydark},
    bodyfont=\normalfont,
    mdframed={
        linewidth=2pt,
        leftline=true, rightline=false, topline=false, bottomline=false,
        linecolor=secondarydark, backgroundcolor=secondarydark!5,
    }
]{secondarydark}

\declaretheoremstyle[
    headfont=\bfseries\color{secondarydark},
    bodyfont=\normalfont,
    mdframed={
        linewidth=2pt,
        leftline=true, rightline=false, topline=false, bottomline=false,
        linecolor=secondarydark
    }
]{secondarydarkline}

\declaretheoremstyle[
    headfont=\bfseries\color{tertiary},
    bodyfont=\normalfont,
    mdframed={
        linewidth=2pt,
        leftline=true, rightline=false, topline=false, bottomline=false,
        linecolor=tertiary, backgroundcolor=tertiary!5,
    }
]{tertiary}

\declaretheoremstyle[
    headfont=\bfseries\color{tertiarylight},
    bodyfont=\normalfont,
    mdframed={
        linewidth=2pt,
        leftline=true, rightline=false, topline=false, bottomline=false,
        linecolor=tertiarylight, backgroundcolor=tertiarylight!5,
    }
]{tertiarylight}

\declaretheoremstyle[
    headfont=\bfseries\color{tertiarydark},
    bodyfont=\normalfont,
    mdframed={
        linewidth=2pt,
        leftline=true, rightline=false, topline=false, bottomline=false,
        linecolor=tertiarydark, backgroundcolor=tertiarydark!5,
    }
]{tertiarydark}

\declaretheoremstyle[
    headfont=\bfseries\color{tertiarydark},
    bodyfont=\normalfont,
    mdframed={
        linewidth=2pt,
        leftline=true, rightline=false, topline=false, bottomline=false,
        linecolor=tertiarydark
    }
]{tertiarydarkline}

\declaretheorem[style=primarydark, numberwithin=section, name=Definicja]{definition}
\declaretheorem[style=primary, numbered=no, name=Inaczej]{intuitive}
\declaretheorem[style=primarylight, numbered=no, name=Notacja]{notation}
\declaretheorem[style=secondarydark, numberwithin=section, name=Twierdzenie]{theorem}
\declaretheorem[style=secondary, numberlike=theorem, name=Wniosek]{lemma}
\declaretheorem[style=secondarylight, numberwithin=section, name=Fakt]{fact}
\declaretheorem[style=tertiary, numbered=no, name=Uwaga]{remark}
\declaretheorem[style=tertiarylight, numbered=no, name=Komentarz]{note}
\declaretheorem[style=tertiarydark, numbered=no, name=Przykład]{example}

\declaretheorem[style=secondarydarkline, name=Dowód]{replacementproof}
\renewenvironment{proof}[1][\proofname]{\vspace{-10pt}\begin{replacementproof}}{\end{replacementproof}}

\declaretheorem[style=tertiarydarkline, name=Dowód]{tmpexplanation}
\newenvironment{explanation}[1][]{\vspace{-10pt}\begin{tmpexplanation}}{\end{tmpexplanation}}