\subsection*{Zadanie 12. (0-6)} 
Treść zadania. MW \\
Funkcja $f$ jest określona wzorem
\begin{align*}
    f(x)= \sin{\frac{29\pi}{6}}x^{\binom{4}{1}}-\frac{1}{\sqrt{3}}\tan{\frac{7\pi}{6}}x^{3}-\frac{\log_{9}{16}}{\frac{1}{2}\log_{3}{2}}x^{2!}+\left(2+1+\frac{1}{2}+\frac{1}{4}+\dots \right)x
\end{align*}
dla każdej liczby dodatniej $x$. \\
\textbf{Pokaż, że daną funkcję da się przedstawić w postaci $\frac{1}{2}x^{4}-\frac{1}{3}x^{3}-4x^{2}+4x$}.
\textbf{Wyznacz największą wartość funkcji $f$ dla $x\in \left(0,1\right)$.}
$\left(\text{Odpowiedź: } f(x)=\frac{95}{96} \right)$