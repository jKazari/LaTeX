% Custom Theorem Styles
\mdfsetup{skipabove=10pt,skipbelow=0pt}

\declaretheoremstyle[
    headfont=\bfseries,
    bodyfont=\normalfont,
    mdframed={
        linewidth=0.7pt,
        leftline=true, rightline=true, topline=true, bottomline=true,
        linecolor=black
    },
    headpunct={\\}
]{box}

\declaretheoremstyle[
    headfont=\bfseries,
    bodyfont=\normalfont,
    mdframed={
        linewidth=0.7pt,
        leftline=true, rightline=false, topline=false, bottomline=false,
        linecolor=black
    },
    headpunct={\\}
]{line}

\declaretheoremstyle[
    headfont=\bfseries,
    bodyfont=\normalfont,
    headpunct={\\}
]{blank}

\declaretheorem[style=box, numberwithin=section, name=Definicja]{definition}
\declaretheorem[style=blank, numbered=no, name=Inaczej]{intuitive}
\declaretheorem[style=blank, numbered=no, name=Notacja]{notation}
\declaretheorem[style=box, numberwithin=section, name=Twierdzenie]{theorem}
\declaretheorem[style=box, numberlike=theorem, name=Wniosek]{lemma}
\declaretheorem[style=box, numberwithin=section, name=Fakt]{fact}
\declaretheorem[style=blank, numbered=no, name=Uwaga]{remark}
\declaretheorem[style=blank, numbered=no, name=Komentarz]{note}
\declaretheorem[style=blank, numbered=no, name=Przykład]{example}

\declaretheorem[style=line, name=Dowód, qed=\qedsymbol]{replacementproof}
\renewenvironment{proof}[1][\proofname]{\begin{replacementproof}}{\end{replacementproof}}

\declaretheorem[style=line, name=Dowód]{tmpexplanation}
\newenvironment{explanation}[1][]{\begin{tmpexplanation}}{\end{tmpexplanation}}