Antek i Basia grywają razem w szachy. Basia w dzieciństwie uczęszczała na kółko szachowe, więc prawdopodobieństwo wygrania przez nią partii jest dwukrotnie większe niż prawdopodobieństwo wygrania przez Antka. Wiedząc, że każdą grę ktoś wygrywa, oblicz prawdopodobieństwo tego, że Antek wygra dokładnie dwa razy w przeciągu pięciu gier.