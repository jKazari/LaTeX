Funkcja $f$ jest określona wzorem
\[
    f(x)= \sin{\frac{29\pi}{6}}x^{\binom{4}{1}}-\frac{1}{\sqrt{3}}\tan{\frac{7\pi}{6}}x^{3}-\frac{\log_{9}{16}}{\frac{1}{2}\log_{3}{2}}x^{2!}+\left(2+1+\frac{1}{2}+\frac{1}{4}+\dots \right)x
\]
dla każdej liczby dodatniej $x$. Pokaż, że daną funkcję da się przedstawić w postaci $\frac{1}{2}x^4 - \frac{1}{3}x^3 - 4x^2 + 4x$. Wyznacz największą wartość funkcji $f$ dla $x\in \left(0,1\right)$.