\usepackage[utf8]{inputenc}
\usepackage[polish]{babel}
\usepackage{amsmath}
\usepackage{amssymb}
\usepackage{fancyhdr}
\usepackage{xcolor}
\usepackage{mdframed}
\usepackage{graphicx}
\usepackage{lastpage}
\usepackage[table,xcdraw]{xcolor}
\usepackage{caption}
\usepackage{subcaption}
\usepackage{float}
\usepackage{graphicx}
\usepackage{hyperref}
% \usepackage{minted}
\usepackage{geometry}
\usepackage{tikz}
\usepackage{xskak}
\usepackage{multicol}
\usepackage{enumitem}
\usepackage{nicefrac}
\usepackage{svg}
\usepackage{amsthm}
\usepackage{gensymb}
\usepackage{titlesec} 

\newcommand{\NUMBER}{1a}
\newcommand{\COURSE}{Naukowe Koło Matematyki\\ Studentów PG}
\newcommand{\TUTOR}{Arkusz\\ Maturalny}
\newcommand{\DEADLINE}{27 kwietnia 2024}
\renewcommand{\familydefault}{\sfdefault}

\newtheorem{tw}{Twierdzenie}
\theoremstyle{definition} 
\newtheorem*{df}{Definicja} 
\newtheorem*{uw}{Uwaga}

\definecolor{PGSilver}{RGB}{200,200,200}
\def\checkmark{\tikz\fill[scale=0.4](0,.35) -- (.25,0) -- (1,.7) -- (.25,.15) -- cycle;} 

\titleformat{\subsection}[runin]{\normalfont\large\bfseries}{\thesubsection}{1em}{}

\newcommand*\Let[2]{\State #1 $\gets$ #2}

\input kvmacros

\geometry{a4paper,left=3cm, right=3cm, top=3cm, bottom=3cm}

\pagestyle {fancy}
\fancyhead[L]{\TUTOR}
\fancyhead[C]{\COURSE}
\fancyhead[R]{\DEADLINE}

\fancyfoot[L]{Poziom rozszerzony}
\fancyfoot[C]{}
\fancyfoot[R]{Strona \thepage /\pageref*{LastPage}}

\def\header#1#2{
	\begin{minipage}[t]{.6\textwidth}
		\vspace*{0pt}
		Egzamin maturalny\hspace{8.4em} Formuła 2023

		\vspace{1.5em}

		\begin{center}
				{\Huge MATEMATYKA}
				\vspace{2em}
				{\Huge Poziom rozszerzony}
		\end{center}
	\end{minipage}
	\hfill
	\noindent
	\begin{minipage}[t]{.4\textwidth}
		\centering
		\vspace*{-\fboxsep}\vspace*{-\fboxrule}
		{\setlength{\fboxrule}{0pt}
		\fbox{\includegraphics[scale=0.2]{figures/logo_samo.png}}
		}
	\end{minipage}
}
\usepackage[nomathsymbols]{polski}