\newcommand{\NUMBER}{1a}

\newcommand{\COURSE}{Naukowe Koło Matematyki\\ Studentów PG}
\newcommand{\TUTOR}{Arkusz\\ Maturalny}
\newcommand{\DEADLINE}{27 kwietnia 2024}

\documentclass[11pt,a4paper]{scrartcl}

\usepackage[utf8]{inputenc}
\usepackage[polish]{babel}
\usepackage{amsmath}
\usepackage{amssymb}
\usepackage{fancyhdr}
\usepackage{color}
\usepackage{mdframed}
\usepackage{graphicx}
\usepackage{lastpage}
\usepackage[table,xcdraw]{xcolor}
\usepackage{caption}
\usepackage{subcaption}
\usepackage{float}
\usepackage{graphicx}
\usepackage{hyperref}
\usepackage{minted}
\usepackage{geometry}
\usepackage{tikz}
\usepackage{xskak}
\usepackage{multicol}
\usepackage{enumitem}
\usepackage{nicefrac}
\usepackage{svg}
\usepackage{amsthm}
\usepackage{gensymb}
\newtheorem{tw}{Twierdzenie}
\theoremstyle{definition} 
\newtheorem*{df}{Definicja} 
\newtheorem*{uw}{Uwaga}
\usepackage{titlesec} 
\definecolor{PGSilver}{RGB}{200,200,200}
\def\checkmark{\tikz\fill[scale=0.4](0,.35) -- (.25,0) -- (1,.7) -- (.25,.15) -- cycle;} 


\titleformat{\subsection}[runin]
  {\normalfont\large\bfseries}{\thesubsection}{1em}{}

\renewcommand{\familydefault}{\sfdefault}

\newcommand*\Let[2]{\State #1 $\gets$ #2}

\input kvmacros


\geometry{a4paper,left=3cm, right=3cm, top=3cm, bottom=3cm}


\pagestyle {fancy}
\fancyhead[L]{\TUTOR}
\fancyhead[C]{\COURSE}
\fancyhead[R]{\DEADLINE}

\fancyfoot[L]{Poziom rozszerzony}
\fancyfoot[C]{}
\fancyfoot[R]{Strona \thepage /\pageref*{LastPage}}


\def\header#1#2{
 
\begin{minipage}[t]{.6\textwidth}
\vspace*{0pt}
Egzamin maturalny\hspace{8.4em} Formuła 2023

\vspace{1.5em}

  \begin{center}
    {\Huge 
MATEMATYKA}
\vspace{2em}

{\Huge Poziom rozszerzony}

  \end{center}
\end{minipage}
\hfill
\noindent
\begin{minipage}[t]{.4\textwidth}
    \centering
    \vspace*{-\fboxsep}\vspace*{-\fboxrule}
    {\setlength{\fboxrule}{0pt}
   \fbox{%
    \includegraphics[scale=0.2]{logo_samo.png}
    }
    }
\end{minipage}

}
\usepackage{polski}


\begin{document}

\header{\NUMBER}{\DEADLINE}

 \subsection*{Zadanie 1. (0-2)}
Aleks posiada beczkę w której znajduje się $200$ litrów soku. W ciągu każdej godziny z beczki tej wyparowuje $\frac{1}{20}$ jej całej zawartości. \textbf{Napisz wzór funkcji $i(t)$, która opisuje ilość soku pozostałego w beczce w zależności od czasu $t$, a następnie wyznacz po ilu godzinach w beczce zostanie mniej niż $75\%$ zawartości soku}.
\subsection*{Zadanie 2. (0-3)}
Antek i Basia grywają razem w szachy. Basia w dzieciństwie uczęszczała na kółko szachowe, więc prawdopodobieństwo wygrania przez nią partii jest dwukrotnie większe niż prawdopodobieństwo wygrania przez Antka. \textbf{Wiedząc, że każdą grę ktoś wygrywa, oblicz prawdopodobieństwo tego, że Antek wygra dokładnie dwa razy w przeciągu pięciu gier}.
\subsection*{Zadanie 3. (0-3)}
\textbf{Wyznacz wszystkie takie styczne do wykresu funkcji $f$ zadanej wzorem $f(x)=\frac{3x^2+14x+3}{x^2+3x+1}$, które są prostopadłe do prostej $x=9$}.
\subsection*{Zadanie 4. (0-3)}
\textbf{Wykaż, że dla każdej liczby rzeczywistej $x>1$ oraz każdej liczby rzeczywistej $y<1$ prawdziwa jest nierówność $x^{2}y^{2}-5xy+x+y+3>0$}. 

\subsection*{Zadanie 5. (0-3)} 
Poprzez $\mathrm{DC}$ i $\mathrm{AB}$ oznaczmy sieczne okręgu, które przecinają się w punkcie $\mathrm{P}$ (zob. rysunek poniżej). \textbf{Niech dane będą następujące wartości: $|\mathrm{AB}| = |\mathrm{BP}| = 8$, $\measuredangle \mathrm{ABC} = 90^\circ$ oraz $|\mathrm{BC}| = 6$. Oblicz pole czworokąta $\mathrm{ABCD}$}.
\begin{figure}[H]
\def\svgwidth{0.55\columnwidth}
\centering
\input{zad5.pdf_tex}
\end{figure}
\subsection*{Zadanie 6. (0-3)}
\textbf{Rozwiąż następujące równanie trygonometryczne}:
$$
 \sin^2(2x) + 4\cos(2x) =4 
$$
\textbf{Zapisz wszystkie obliczenia}.

\subsection*{Zadanie 7. (0-4)}
Niech dany będzie taki ostrosłup prawidłowy sześciokątny, że jego pole powierzchni bocznej jest dwukrotnie większe od jego pola podstawy. \textbf{Wyznacz wartość $\tan^{2}(\alpha)$ kąta $\alpha$ zawartego między sąsiednimi ścianami bocznymi tego ostrosłupa.}

\subsection*{Zadanie 8. (0-4)}
Niech dany będzie czworokąt $\mathrm{ABCD}$ o następujących długościach boków: $|\mathrm{AD}|=3$, $|\mathrm{CD}|=3\sqrt{3}$, $|\mathrm{BC}|=2\sqrt{2}-\sqrt{3}$. Czworokąt ten dodatkowo można wpisać w okrąg. \textbf{Oblicz długość boku $|\mathrm{AB}|$ i przedstaw tę długość w postaci liczby naturalnej, wiedząc, że kąt pomiędzy bokami $|\mathrm{AD}|$ i $|\mathrm{DC}|$ wynosi $30\degree$}.

\subsection*{Zadanie 9. (0-4)}
\textbf{Rozwiąż poniższą nierówność:}
\begin{align*}
\sqrt{4x^{2}-12x+9}\leq 1+\sqrt{9x^{2}-12x+4}
\end{align*}
\emph{Wskazówka: skorzystaj z tego, że $\sqrt{a^{2}}=|a|$ dla każdej liczby rzeczywistej $a$.}


\subsection*{Zadanie 10. (0-4)}
Budujemy nieskończony ciąg figur $\{F_n\}_{n=1}^\infty$.
Figura $F_1$ jest kwadratem o boku długości $a=a_{1}$.
Kolejne figury $F_n$ są budowane według następujących reguł:
\begin{itemize}
    \item Podczas budowy figury $F_{n}$ doklejamy kolejne kwadraty \underline{jedynie} do kwadratów, które zostały dodane w kroku numer ($n-1$). 
    \item Dzielimy boki figury $F_{n-1}$ na trzy równe części.
    \item Do środkowej części każdego z podzielonych boków doklejamy mniejszy kwadrat o boku długości $\frac{a_{n-1}}{3}$.
\end{itemize}
(zob. rysunek poniżej).

\begin{figure}[h!] 
\def\svgwidth{0.25\columnwidth} 
\label{maps}
\input{zad101.pdf_tex}
    \qquad
    \centering
    \subfloat{{\def\svgwidth{0.4\columnwidth}
\input{zad102.pdf_tex}}}%
    \subfloat{{\subfloat{{\def\svgwidth{0.4\columnwidth}
\input{zad103.pdf_tex}}}}}%
\label{maps}
\end{figure}
\begin{enumerate}
    \item \textbf{Wykaż, że pole $P_{n}$ figury $F_n$ wyraża się następującym wzorem}:
$$
P_n = a^2\left({5\over 3} - {2 \over {3^n}}\right)
$$
\item \textbf{Wyznacz wszystkie wartości liczby $a$, dla której pole figury $F_{\infty}$ jest mniejsze od $\frac{1}{3}$.}

\end{enumerate}












\subsection*{Zadanie 11. (0-5)}
\textbf{Wyznacz wszystkie wartości parametru rzeczywistego $m$, dla którego równanie:}
\begin{align*}
        (m+2)x^{2}+3mx+m-2=0
\end{align*}
\textbf{ma dwa różne rozwiązania rzeczywiste $x_{1}$ oraz $x_{2}$ spełniające warunek $x_{1}^{3}+x_{2}^{3}>1$}. \\

$\left(\text{Odpowiedź: }m\in \left(-\infty,\frac{11-3\sqrt{21}}{17} \right)\cup\left(\frac{11+3\sqrt{21}}{17} ,\infty\right) \right)$

\subsection*{Zadanie 12. (0-6)} 
Treść zadania. MW \\
Funkcja $f$ jest określona wzorem
\begin{align*}
    f(x)= \sin{\frac{29\pi}{6}}x^{\binom{4}{1}}-\frac{1}{\sqrt{3}}\tan{\frac{7\pi}{6}}x^{3}-\frac{\log_{9}{16}}{\frac{1}{2}\log_{3}{2}}x^{2!}+\left(2+1+\frac{1}{2}+\frac{1}{4}+\dots \right)x
\end{align*}
dla każdej liczby dodatniej $x$. \\
\textbf{Pokaż, że daną funkcję można przedstawić w postaci $f(x)=\frac{1}{2}x^{4}-\frac{1}{3}x^{3}-4x^{2}+4x$}.
\textbf{Wyznacz największą wartość funkcji $f$ dla $x\in \left(0,1\right)$.}
$\left(\text{Odpowiedź: } f(x)=\frac{95}{96} \right)$


\subsection*{Zadanie 13. (0-6)}
W kartezjańskim układzie współrzędnych kreślimy dwie parabole:
\begin{itemize}
    \item $f(x) = -\frac{1}{4}x^2-2$
    \item $g(x) = \frac{1}{8}x^2 + \frac{3}{2}$
\end{itemize}
Prosta $h$ przecina obie parabole w punktach $A$ oraz $B$ (rysunek poniżej). Wiemy, że tangens kąta nachylenia prostej $h$ wynosi 2, i że odcięta punktu $A$ równa się $0$. Punkty $C$ i $D$ są równo odległe od punktów $A$ i $B$ oraz ich odległość od prostej $h$ wynosi $\sqrt{5}$. 

\begin{figure}[H]
\def\svgwidth{0.55\columnwidth}
\centering
\input{zad13.pdf_tex}
\end{figure}

\begin{enumerate}
    \item \textbf{Wyznacz współrzędne punktów $C$ i $D$}. (odp. $C=(-1,1), D=(3,-1)$)
    \item \textbf{Wyznacz promień okręgu wpisanego w czworokąt $ABCD$}. (odp. $r=\frac{\sqrt{10}}{2}$)

\end{enumerate}

\end{document}
