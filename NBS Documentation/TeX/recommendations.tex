\section{Recommendations}

\subsection{Continue Data Profiling}
Data profiling must be an ongoing process to identify and resolve issues across all tables. Each table should be individually inspected, and any issues should be documented. For instance, the \texttt{tables\_inventory.json} file can be extended to include detailed issue summaries for each table. For example:

\begin{lstlisting}[language=json]
[
    {
        "display-name": "NBS Test Order",
        "logical-name": "cr675_testorder",
        "set-name": "cr675_testorders",
        "primary-id": "cr675_testorderid",
        "primary-name": "cr675_orderid",
        "attribute-count": 90,
        "custom-attribute-count": 72,
        "row-count": 3394,
        "issues-summary": [
            "Issue 1",
            "Issue 2",
            "Issue 3"
            // More issues...
        ],
        "attributes": [
            {
                "display-name": "Client",
                "logical-name": "cr675_client",
                "description": "cr675_testorder - cr675_client",
                "data-type": "Lookup",
                "is-custom": true,
                "statistics": {
                    "non-null-count": 3394,
                    "unique-values-count": 9,
                    "top-5-most-frequent-values": {
                        "NFM_P CC205 RDB Ballistic": 1161,
                        "NFM_P CC107 QC": 1143,
                        "NFM_P CC208 RDB Strategic": 844,
                        "NFM_P CC210 HJELM": 147,
                        "NFM US LLC": 60
                    }
                },
                "issues": [
                    "Issue 1",
                    "Issue 2",
                    "Issue 3"
                    // More issues...
                ]
            }
            // More columns... 
        ],
        "relationships": [
            {
                "related-table": "systemuser",
                "relationship_type": "many-to-one",
                "is-custom-relationship": false,
                "local-attribute": "createdby",
                "foreign-attribute": "systemuserid"
            }
            // More relationships...
        ]
    }
    // More tables...
]
\end{lstlisting}

\subsection{Migrate NBS System from Default Environment}
The NBS system should be migrated from the \texttt{NFM Group Def} environment to the dedicated \texttt{NFM Group NBS DEV} environment. This migration will improve data management, security, and scalability, ensuring a more stable and efficient system.

\subsection{Refactor and Standardize Data}
Tables and columns with incorrect data types should be refactored. Standardization should ensure that all columns are stored in appropriate formats, facilitating easier and more reliable data analysis.

\subsection{Consolidate Redundant Tables and Columns}
Redundant tables and columns should be merged or removed to simplify the data model. Consolidating overlapping tables will reduce complexity and improve system performance.

\subsection{Implement Required Fields and Validation}
Critical columns should be designated as "required" to enhance data completeness. Additionally, validation rules should be enforced to ensure consistent data entry, especially for numeric values, dates, and key identifiers.

\subsection{Remove Unused Input Fields}
Unused input fields should be removed as they do not contribute to the system's functionality. An audit of all current input fields is recommended to identify and eliminate those that are not in use. This will streamline the user interface and improve system performance by reducing redundant metadata.

\subsection{Clean Mostly Filled Dictionaries}
Many dictionaries in the system, such as those for threats, clients, or ammunition, are largely filled but suffer from inconsistent or incomplete data entries. These dictionaries should be reviewed and cleaned. Afterward, access restrictions should be implemented to control who can add new entries. Additionally, strict data validation rules should be enforced during entry.

\subsection{Use External Registers Where Possible}
Where possible, external registers or reference data should be utilized to reduce data redundancy. For example, project lists or other externally maintained registers can be linked directly to the NBS system, improving data reliability and consistency.

\newpage