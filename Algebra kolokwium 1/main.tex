\documentclass[a4paper,12pt]{article}
% Math essential packages
\usepackage[dvipsnames]{xcolor}
\usepackage{latexsym, mathtools}
\usepackage{amsmath, amssymb, amsfonts, amsthm, amsxtra}
\usepackage[nomathsymbols]{polski}
\usepackage{amscd, tikz-cd}
\usepackage[skip=10pt, indent=0pt]{parskip}
\usepackage[a4paper, left=30mm, right=30mm, top=25mm, bottom=25mm]{geometry}
\usepackage{graphicx, float}
\usepackage[most,many,breakable]{tcolorbox}
\usepackage{relsize}
\usepackage{fancyhdr}
\usepackage{url}
\usepackage[colorlinks=true,citecolor=Periwinkle,urlcolor=Periwinkle,linkcolor=Periwinkle,pdfpagemode=UseNone]{hyperref}

\usepackage[framemethod=TikZ]{mdframed}
\usepackage{thmtools}
% Custom Theorem Styles
\mdfsetup{skipabove=10pt,skipbelow=0pt}

\declaretheoremstyle[
    headfont=\bfseries,
    bodyfont=\normalfont,
    mdframed={
        linewidth=0.7pt,
        leftline=true, rightline=true, topline=true, bottomline=true,
        linecolor=black
    },
    headpunct={\\}
]{box}

\declaretheoremstyle[
    headfont=\bfseries,
    bodyfont=\normalfont,
    mdframed={
        linewidth=0.7pt,
        leftline=true, rightline=false, topline=false, bottomline=false,
        linecolor=black
    },
    headpunct={\\}
]{line}

\declaretheoremstyle[
    headfont=\bfseries,
    bodyfont=\normalfont,
    headpunct={\\}
]{blank}

\declaretheorem[style=box, numberwithin=section, name=Definicja]{definition}
\declaretheorem[style=blank, numbered=no, name=Inaczej]{intuitive}
\declaretheorem[style=blank, numbered=no, name=Notacja]{notation}
\declaretheorem[style=box, numberwithin=section, name=Twierdzenie]{theorem}
\declaretheorem[style=box, numberlike=theorem, name=Wniosek]{lemma}
\declaretheorem[style=box, numberwithin=section, name=Fakt]{fact}
\declaretheorem[style=blank, numbered=no, name=Uwaga]{remark}
\declaretheorem[style=blank, numbered=no, name=Komentarz]{note}
\declaretheorem[style=blank, numbered=no, name=Przykład]{example}

\declaretheorem[style=line, name=Dowód, qed=\qedsymbol]{replacementproof}
\renewenvironment{proof}[1][\proofname]{\begin{replacementproof}}{\end{replacementproof}}

\declaretheorem[style=line, name=Dowód]{tmpexplanation}
\newenvironment{explanation}[1][]{\begin{tmpexplanation}}{\end{tmpexplanation}}

\usepackage{XCharter}

\title{Algebra liniowa - Kolokwium I}
\author{Zachariasz Jażdżewski}
% \date{}

\begin{document}
\maketitle

%----Proper document------------------------------------------------------------

\section{Definicje}
\hfill

\begin{definition}[Przestrzeń liniowa]
	System algebraiczny $(V,K,+,\cdot)$, gdzie $V$ jest niepustym zbiorem wektorów, $K$ jest ciałem, $+$ jest dodawaniem wektorów, a $\cdot$ jest mnożeniem wektorów przez skalary oraz spełnione są warunki:
	\begin{enumerate} 
		\item $\forall_{x,y \in V}\quad x+y = y+x$
		\item $\forall_{x,y,z \in V}\quad x+(y+z) = (y+x)+z$
		\item $\exists_{0 \in V}\, \forall_{x \in V}\quad x+0=x$
		\item $\forall_{x \in V}\, \exists_{-x \in V}\quad x+(-x) = 0$
		\item $\forall_{\alpha \in K}\, \forall_{x,y \in V}\quad \alpha(x+y) = \alpha x + \alpha y$
		\item $\forall_{\alpha, \beta \in K}\, \forall_{x \in V}\quad (\alpha + \beta )x = \alpha x + \beta x$
		\item $\forall_{\alpha, \beta \in K}\, \forall_{x \in V}\quad \alpha (\beta x) = (\alpha \beta)x$
		\item $\forall_{x \in V}\quad 1x = x$, gdzie 1 to jedynka ciała $K$.   
	\end{enumerate}
\end{definition}

\begin{definition}[Podprzestrzeń liniowa]
	Podprzestrzenią liniową \\ przestrzeni $(V,K,+,\cdot)$ nazywamy przestrzeń $(W,K,+,\cdot)$ gdzie $W$ jest podzbiorem $V$. 
\end{definition}

\begin{definition}[Kombinacja liniowa wektorów]
	Kombinacją liniową wektorów $v_1, v_2, \dots, v_n$ o współczynnikach $\alpha_1, \alpha_2, \dots, \alpha_n$ nazywamy wektor
	\[
		v = \sum_{i=1}^{n} \alpha_i v_i 
	\]  
\end{definition}

\begin{example}
	Kombinacją liniową wektorów $u_1 = (1,2,3)$ i $u_2 = (4,5,6)$ jest wektor
	\[
		v = 5u_1 - 3u_2 = 5(1,2,3) - 3(4,5,6) = (5,10,15) - (12,15,18) = (-7,-5,-3)
	\]
\end{example}

\newpage

\begin{definition}[Liniowo niezależny układ wektorów]
	Układ wektorów \\ $(v_1,v_2, \dots, v_n)$ jest liniowo niezależny, jeśli równanie wektorowe
	\[
		x_1 v_1 + x_2 v_2 + \dots + x_n v_n = 0
 	\] 
	ma \emph{tylko zerowe rozwiązanie}.
\end{definition}

\begin{intuitive}
	\[
		x_1 v_1 + x_2 v_2 + \dots + x_n v_n = 0 \iff (x_1, x_2, \dots, x_n) = (0,0,\dots,0)
	\]
\end{intuitive}

\begin{definition}[Liniowo zależny układ wektorów]
	Układ wektorów jest liniowo zależny, jeśli nie jest on liniowo niezależny, czyli gdy równanie wektorowe
	\[
		x_1 v_1 + x_2 v_2 + \dots + x_n v_n = 0
	\]
	ma rozwiązanie niezerowe.
\end{definition}

\begin{intuitive}
	Układ jest liniowo zależny, jeśli istnieją skalary $x_1, x_2, \dots, x_n$ nie wszystkie równe zeru, takie, że spełnione jest równanie powyżej.
\end{intuitive}

\begin{definition}[Baza przestrzeni]
	Bazą przestrzeni $V$ jest układ  \\ $B = (v_1, v_2, \dots, v_n)$ wektorów z przestrzeni $V$, który
	\begin{enumerate}
		\item Jest liniowo niezależny
		\item Generuje przestrzeń $B$, czyli $\mathcal{L}(B) = V$ 
	\end{enumerate}
\end{definition}

\begin{definition}[Wymiar przestrzeni $V$]
	Wymiar przestrzeni $V$ to ilość elementów dowolnej bazy przestrzeni wektorowej $V$. Jeśli przestrzeń $V$ nie jest skończenie wymiarowa, to jest ona nieskończenie wymiarowa. 
\end{definition}

\begin{notation}
	Wymiar przestrzeni $V$ zapisujemy jako $\dim V$ od angielskiego "dimension".
\end{notation}

\begin{example}
	.
	\begin{itemize}
		\item Przestrzeń $\R_3$ jest $3$-wymiarowa. 
		\item Przestrzeń $\R_n[x]$ wielomianów stopnia co najwyżej $n$-tego jest wymiaru $n+1$. 
		\item Przestrzeń $V = \{ 0 \}$ jest $0$-wymiarowa.
	\end{itemize}
\end{example}

\newpage

\begin{definition}[Wektor współrzędnych]
	Niech $v$ będzie wektorem przestrzeni $V$ generowanej przez bazę $B = (b_1,b_2,\dots,b_n)$.
	\[
		v = r_1 b_1 + r_2 b_2 + \dots + r_n b_n
	\]
	Wektorem współrzędnych wektora $v$ względem bazy $B$ nazywamy wektor
	\[
		[v]_B = 
		\begin{bmatrix}
			 \ r_1\  \\
			 r_2 \\
			 \vdots \\
			 r_n \\
		\end{bmatrix}
	\]
	skalary $r_1, r_2, \dots, r_n$ nazywamy współrzędnymi wektora $v$ względem bazy $B$.  
\end{definition}

\begin{definition}[Izomorfizm przestrzeni]
	Izomorfizm przestrzeni $V$ na przestrzeń $W$ to przekształcenie $\varphi\colon V \to W$ spełniające warunki:
	\begin{enumerate}
		\item $\varphi$ jest różnowartościowe i $\varphi(V) = W$
		\item $\forall_{x,y \in V}\quad \varphi(x+y) = \varphi(x) + \varphi(y)$
		\item $\forall_{x \in V}\, \forall_{\alpha \in K}\quad \varphi(\alpha \cdot x) = \alpha \cdot \varphi (x)$    
	\end{enumerate} 
\end{definition}

\begin{definition}[Macierz przejścia od bazy do bazy]
	Macierzą przejścia z bazy $B = (b_1,b_2,\dots,b_3)$ do bazy $C = (c_1,c_2,\dots,c_n)$ przestrzeni wektorowej $V$ jest
	\[
		P^B_C = 
		\begin{bmatrix}
			| & | &  & |  \\
			[b_1]_C & [b_2]_C & \dots & [b_n]_C  \\
			| & | &  & |  \\
		\end{bmatrix}
	\] 
\end{definition}

\begin{intuitive}
	Macierz przejścia z bazy $B$ do bazy $C$ to macierz w której kolumny to wektory współrzędnych elementów $b_1,b_2,\dots,b_n$ bazy $B$ względem bazy $C$.
\end{intuitive}

\begin{definition}[Przekształcenie liniowe]
	Przekształcenie liniowe przestrzeni $V$ w przestrzeń $W$ to funkcja $T\colon V \to W$ spełniająca warunki:
	\begin{enumerate}
		\item $T(x+y) = T(x) + T(y)$ \hfill (addytywność przekształcenia)
		\item $T(\alpha \cdot x) = \alpha \cdot T(x)$ \hfill (jednorodność przekształcenia)
	\end{enumerate}
\end{definition}

\begin{remark}
	Warunki 1. i 2. są równoznaczne z poniższym warunkiem
	\[
		T(\alpha x + \beta y) = \alpha T(x) + \beta T(y)
	\]
\end{remark}

\newpage

\begin{definition}[Obraz przekształcenia]
	Obraz przekształcenia $T$ to zbiór 
	\[
		\Image T = T(V) = \{ T(x) : x \in V \}
	\]
\end{definition}

\begin{intuitive}
	Obrazem przekształcenia jest zbiór wszystkich wektorów z przestrzeni $V$ przekształconych w przestrzeń $W$ za pomocą przekształcenia $T$.
\end{intuitive}

\begin{notation}
	Obraz przekształcenia $T$ oznaczamy przez $\Image T$ od angielskiego "image".
\end{notation}

\begin{definition}[Jądro przekształcenia]
	Jądro przekształcenia $T$ to zbiór
	\[
		\Ker T = T^{-1}(\{0\}) = \{ x \in V : T(x) = 0 \}  
	\]
\end{definition}

\begin{intuitive}
	Jądrem przekształcenia jest zbiór takich wektorów przestrzeni $V$, które po przekształceniu dają wektor zerowy przestrzeni $W$.
\end{intuitive}

\begin{notation}
	Jądro przekształcenia $T$ oznaczamy przez $\Ker T$ od angielskiego "kernel".
\end{notation}

\begin{definition}[Monomorfizm]
	Przekształcenie $T\colon V \to W$ jest monomorfizmem, gdy każdy wektor z przestrzeni $W$ jest obrazem co najwyżej jednego wektora z przestrzeni $V$.  
\end{definition}

\begin{intuitive}
	Przekształcenie jest monomorfizmem, jeśli jest różnowartościowe (jest iniekcją).
	\[
		\forall_{x,x' \in V}\quad T(x) = T(x') \implies x = x'
	\]
\end{intuitive}

\begin{definition}[Epimorfizm]
	Przekształcenie $T\colon V \to W$ jest epimorfizmem, gdy każdy wektor z przestrzeni $W$ jest obrazem co najmniej jednego wektora z przestrzeni $V$.  
\end{definition}

\begin{intuitive}
	Przekształcenie jest epimorfizmem, jeśli jest funkcją "na" (jest suriekcją).
	\[
		\forall_{y \in W}\, \exists_{x \in V}\quad y = T(x)
	\]
\end{intuitive}

\begin{definition}[Macierz przekształcenia liniowego]
	Macierzą przekształcenia liniowego $T\colon V \to W$ z przestrzeni $V$ o bazie $B = (b_1,b_2,\dots,b_n)$ w przestrzeń $W$ o bazie $C = (c_1,c_2,\dots,c_n)$ jest
	\[
		[T]^B_C = 
		\begin{bmatrix}
			| &  &  | \\
			[T(b_1)]_C & \dots & [T(b_n)]_C  \\
			| &  & |  \\
		\end{bmatrix}
	\]   
\end{definition}

\begin{intuitive}
	Macierz przekształcenia liniowego $T$ względem baz $B$ i $C$ to macierz w której kolumny to wektory $C$-współrzędnych wektorów $T(b_1), \dots, T(b_n)$. 
\end{intuitive}

\newpage

\section{Twierdzenia}
\hfill

\begin{theorem}[Twierdzenie wymiarowe]
	Jeśli $T\colon V \to W$ jest przekształceniem liniowym i $V$ jest przestrzenią skończonego wymiaru, to
	\[
		\dim \Ker T + \dim \Image T = \dim V
	\] 
\end{theorem}
\begin{proof}
	Niech $(a_1,\dots,a_r)$ i $(b_1,\dots,b_p)$ będą odpowiednio bazami przestrzeni $\Ker T$ i $\Image T$. Niech $b_i' \in V$ będą wektorami takimi, że $T(b_i') = b_j$ dla $j \in \{1,\dots,p\}$.
	
	Do udowodnienia twierdzenia wymiarowego musimy udowodnić, że \\ $B = (a_1,\dots,a_r,b_1',\dots,b_p')$ jest bazą przestrzeni $V$. Aby to zrobić musimy wykazać, że układ $B$ jest liniowo niezależny. Bierzemy zatem pod uwagę kombinację liniową
	\[
		\sum_{i=1}^{r} x_i a_i + \sum_{j=1}^{p} y_j b_j' = 0
	\]
	Musimy pokazać, że wszystkie $x_i$ i wszystkie $y_j$ są równe $0$.
	\[
		0 
		\overset{1}{=} 
		T(0) 
		\overset{2}{=} 
		T\left( \sum_{i=1}^{r} x_i a_i + \sum_{j=1}^{p} y_j b_j' \right) 
		\overset{3}{=}
		\sum_{i=1}^{r} x_i T(a_i) + \sum_{j=1}^{p} y_j T(b_j') 
		\overset{4}{=}
		\sum_{j=1}^{p} y_j b_j
	\]
	\begin{note}
		\hfill
		\begin{enumerate}
			\item Wynika z liniowości przekształcenia $T$
			\item Wektor zerowy możemy przedstawić jako jakąś kombinację liniową wektorów przestrzeni $V$
			\item Wyciągamy skalary przed przekształcenie $T(a_i)$ i $T(b_j')$
			\item Jako, że $a_i$ sa elementami bazy jądra $T$, to po przekształceniu się zerują. Do tego z założenia $T(b_j') = b_j$ 
		\end{enumerate}
	\end{note}
	zatem 
	\[
		\sum_{j=1}^{p} y_j b_j = 0
	\]
	Jako, że wektory $b_j$ są liniowo niezależne, więc z powyższej równości otrzymujemy, że skalary $y_j$ są równe $0$.

	Stąd, podstawiając do oryginalnej kombinacji liniowej otrzymujemy
	\[
		\sum_{i=1}^{r} x_i a_i = 0
	\]

	Z tej zaś równości i z liniowej niezależności wektorów $a_i$ wynika, że skalary $x_i$ są równe $0$.

	Teraz udowodnimy, że każdy wektor $v \in V$ jest kombinacją liniową wektorów układu $B$.

	Przede wszystkim, ponieważ wektory $b_j$ generują przestrzeń $T(V)$ i $T(v) \in T(V)$, to istnieją skalary $y_j$ takie, że 
	\[
		T(v) = \sum_{j=1}^{p} y_j b_j
	\]
	Zauważmy teraz, że wektor 
	\[
		v - \sum_{j=1}^{p} y_j b_j' \in \Ker T
	\]
	więc istnieją skalary $x_i$ takie, że 
	\[
		v - \sum_{j=1}^{p} y_j b_j' = \sum_{i=1}^{r} x_i a_i
	\]
	i dlatego 
	\[
		v = \sum_{i=1}^{r} x_i a_i + \sum_{j=1}^{p} y_j b_j'
	\]
	Z powyższego wynika, że zbiór $B$ jest bazą przestrzeni $V$ i dlatego
	\[
		\dim V = |B| = r + p = \dim \Ker T + \dim \Image T
	\] 
\end{proof}

\begin{theorem}
	Obraz $\Image T$ przekształcenia liniowego $T\colon V \to W$ jest podprzestrzenią przestrzeni $W$.
\end{theorem}
\begin{proof}
	Zbiór $T(V)$ jest niepusty, bo $0 \in V$ i $0 - T(0) \in T(V)$. Weźmy teraz dowolne wektory $y, y' \in T(V)$ i skalary $\alpha, \beta \in K$. Wystarczy pokazać, że 
	\[
		\alpha y + \beta y' \in T(V)
	\] 
	Niech $x, x' \in V$ będą takie, że $T(x) = y$ i $T(x') = y'$. Wtedy $\alpha x + \beta y' \in V$. Stąd i z liniowości przekształcenia $T$ wynika, że
	\[
		\alpha y + \beta y' = \alpha T(x) + \beta T(x') = T(\alpha x + \beta x') \in T(V)
	\]  
\end{proof}

\newpage

\begin{theorem}
	Jądro $\Ker T$ przekształcenia liniowego $T\colon V \to W$ jest podprzestrzenią przestrzeni $W$. 
\end{theorem}
\begin{proof}
	Ponieważ $T(0) = 0 \in \{0\}$, więc $0 \in \Ker T$ i dlatego zbiór $\Ker T$ jest niepusty. Niech $x,x' \in \Ker T$ i $\alpha, \beta \in K$. Do dowodu twierdzenia wystarczy pokazać, że $\alpha x + \beta x' \in \Ker T$. Ponieważ $x,x' \in \Ker T$, więc $T(x), T(x') \in \{0\}$ i dlatego $\alpha T(x) + \beta T(x') \in \{0\}$. Stąd i z liniowości przekształcenia $T$ wynika, że
	\[
		T(\alpha x + \beta x') \in \{0\} 
	\]
	zatem
	\[
		\alpha x + \beta x' \in \Ker T
	\]
\end{proof}

\begin{theorem}
	Addytywność (1.) i jednorodność (2.) przekształcenia \\
	liniowego są równoznaczne z 
	\[
		T(\alpha x + \beta y) = \alpha T(x) + \beta T(y)
	\]
\end{theorem}
\begin{proof}
	\[
		T(\alpha x + \beta y) \overset{1}{=} T(\alpha x) + T(\beta y) \overset{2}{=} \alpha T(x) + \beta T(y) 
	\]
\end{proof}

%-------------------------------------------------------------------------------

\end{document}
