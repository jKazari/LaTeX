% Math essential packages
\usepackage{latexsym, mathtools}
\usepackage{amsmath, amssymb, amsfonts, amsthm, amsxtra}
\usepackage{xfrac}
\usepackage[nomathsymbols]{polski}
\usepackage{amscd, tikz-cd, pgfplots}
\usepackage{subcaption}
\usepackage[skip=10pt, indent=0pt]{parskip}
\usepackage[a4paper, left=30mm, right=30mm, top=25mm, bottom=25mm]{geometry}
\usepackage{graphicx, float}
\usepackage{svg}
\usepackage{xcolor}
\usepackage[most,many,breakable]{tcolorbox}
\usepackage{relsize}
\usepackage{fancyhdr}
\usepackage{url}
\usepackage[colorlinks=true,citecolor=blue,urlcolor=blue,linkcolor=blue,pdfpagemode=UseNone]{hyperref}

\usepackage[framemethod=TikZ]{mdframed}
\usepackage{thmtools}
\usetikzlibrary{calc}

% Number sets
\newcommand{\N}{\mathbb{N}}
\newcommand{\Z}{\mathbb{Z}}
\newcommand{\Q}{\mathbb{Q}}
\newcommand{\R}{\mathbb{R}}
\newcommand{\C}{\mathbb{C}}
% \newcommand{\emph}[1]{\textit{#1}}

% Custom Theorem Styles
\mdfsetup{skipabove=10pt,skipbelow=0pt}

\declaretheoremstyle[
    headfont=\bfseries,
    bodyfont=\normalfont,
    mdframed={
        linewidth=0.7pt,
        leftline=true, rightline=true, topline=true, bottomline=true,
        linecolor=black
    }
]{box}

\declaretheoremstyle[
    headfont=\bfseries,
    bodyfont=\normalfont,
    mdframed={
        linewidth=0.7pt,
        leftline=true, rightline=false, topline=false, bottomline=false,
        linecolor=black
    }
]{line}

\declaretheoremstyle[
    headfont=\bfseries,
    bodyfont=\normalfont,
]{normal}

\declaretheorem[style=box, numberwithin=section, name=Definicja]{definition}
\declaretheorem[style=normal, numbered=no, name=Inaczej]{intuitive}
\declaretheorem[style=normal, numbered=no, name=Notacja]{notation}
\declaretheorem[style=box, numbered=no, name=Twierdzenie]{theorem}
\declaretheorem[style=box, numbered=no, name=Wniosek]{lemma}
\declaretheorem[style=box, numberwithin=section, name=Fakt]{fact}
\declaretheorem[style=normal, numbered=no, name=Uwaga]{remark}
\declaretheorem[style=normal, numbered=no, name=Komentarz]{note}
\declaretheorem[style=normal, numbered=no, name=Przykład]{example}

\declaretheorem[style=line, name=Dowód, qed=\qedsymbol]{replacementproof}
\renewenvironment{proof}[1][\proofname]{\begin{replacementproof}}{\end{replacementproof}}

\declaretheorem[style=line, name=Dowód]{tmpexplanation}
\newenvironment{explanation}[1][]{\begin{tmpexplanation}}{\end{tmpexplanation}}